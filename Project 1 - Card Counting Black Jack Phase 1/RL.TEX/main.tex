\documentclass{article}
\usepackage[utf8]{inputenc}
\usepackage{booktabs}
\usepackage{geometry}
 \geometry{
 a4paper,
 total={170mm,257mm},
 left=20mm,
 top=20mm,
 }
 \usepackage{graphicx}
\usepackage{titling}
\usepackage{hyperref}
\usepackage{xcolor}
\usepackage{array}

\title{ Card Counting Black Jack: Phase 1}
\date{23/5/2025}

 
 \usepackage{fancyhdr}
\fancypagestyle{plain}{%  the preset of fancyhdr 
    \fancyhf{} % clear all header and footer fields
    \fancyhead[L]{Reinforcement Learning and Dynamic Optimization}
    \fancyhead[R]{\thedate}
    \fancyfoot[C]{\thepage}
}
\makeatletter
\def\@maketitle{%
  \newpage
  \null
  \vskip 1em%
  \begin{center}%
  \let \footnote \thanks
    {\LARGE \@title \par}%
    \vskip 1em%
    %{\large \@date}%
  \end{center}%
  \par
  \vskip 1em}
\makeatother

\usepackage{lipsum}
\usepackage{amsmath}
\usepackage{cmbright}

\begin{document}

\maketitle

\noindent\begin{tabular}{@{}ll}
       Michalis Lamprakis & 2020030077\\
       Dimitris Ilia & 2020030200
\end{tabular}

\section*{Project context}

 The goal of this problem is to try to find an optimal Black Jack 
 policy that beats the dealer in a single player game (agent against 
 the dealer). A simplified version of Blackjack is simulated 
 without betting, "card splitting" or other more advanced rules.
 The project is structured in two main phases: i) 
 learning a basic policy without card counting and ii) 
 extending the agent to exploit card counting using the "Hi-Lo" 
 system. 

\section*{Implementantion}

The whole Implementantion is based on {\bf BlackJackEnv} class that
we made and contains all the necessary functions to simulate 
the game. The provided code contains detailed comments.\\

\noindent In {\bf task 0} a simple game setup is implemented where the user can play
against the dealer with a simple visualization enviroment(console prints).
As a baseline, we evaluated a completely random policy and a 
simple threshold one where  the agent hits if the player’s hand is 
below 17, and sticks otherwise. For a number of 100000 games, 
we observe the following results:

\begin{table}[h!]
\centering
\begin{tabular}{lccc}
\toprule
\textbf{Policy} & \textbf{Win} & \textbf{Draw} & \textbf{Lose} \\
\midrule
Random (Uniform)     & 28.42\% & 4.25\%  & 67.33\% \\
Threshold (Stick $\geq$ 17) & 40.99\% & 10.14\% & 48.88\% \\
\bottomrule
\end{tabular}
\end{table}

\noindent As regards the threshold policy, 
while it is better than a purely random strategy, still underperforms 
compared to our learned Q-agent, as we will see below.
The threshold policy fails to incorporate the dealer’s upcard or ace usability, 
which are critical in optimal decision-making.\\


\noindent For {\bf task 1} we train an agent to find an optimal
action policy using tabular Q-learning with the following 
update formula:
\[
    Q(s,a) \leftarrow Q(s,a) + \alpha \left( r + \gamma \max_{a'} Q(s',a') - Q(s,a) \right)
\]

\noindent After experimenting with different values we 
decided to use {\bf $\alpha=0.1$} (avoids too fast or too slow convergence)
and {\bf $\gamma=1.0$} (because future rewards are just as valuable as immediate ones).
Also for the exploration rate we satarted with {\bf $\epsilon=1.0$} and
we decay it up to {\bf $\epsilon=0.05$}, linearly over episodes.\\

\noindent We model the game using a standard reinforcement learning framework.
The state space for this environment is the tuple $(player hand total,dealer upcard,usable ace)$.
After training the agent for 500000 episodes and playing
100000 games with the learned policy
(large enough numbers for the action space, the agent’s win rate stabilizes) 
we observe a win rate of $\approx 42\%$, a loss rate of $\approx 48\%$
and $\approx 9\%$ rate for draw.\\

\noindent After searching online ,there is a proven theory
for the optimal strategy of BlackJack (without card counting in this case)
and using this as a benchmark we compare our learned policy
and observe a mismatch rate of less than 10\% which is 
acceptable. We are indeed convinced that our agent
has learned an optimal policy, but even with this
the loss rate is higher than the win rate (so we do not yet beat the
casinos). 

\newpage

\noindent This is the strategy we used as a reference:\\

\noindent
\begin{minipage}[t]{0.48\textwidth}
\colorbox{blue!20}{\textbf{Hard Totals (No usable ace)}}\\[1ex]
\begin{tabular}{|>{\raggedright\arraybackslash}p{2cm}|>{\raggedright\arraybackslash}p{3cm}|>{\raggedright\arraybackslash}p{2cm}|}
\hline
\textbf{Player Total} & \textbf{Dealer Showing} & \textbf{Action} \\
\hline
  17--21 & Any & STICK \\
  13--16 & 2--6 & STICK \\
  13--16 & 7--A & HIT \\
  12 & 4--6 & STICK \\
  12 & 2--3 or 7--A & HIT \\
  5--11 & Any & HIT \\
\hline
\end{tabular}
\end{minipage}%
\hfill
\begin{minipage}[t]{0.48\textwidth}
\colorbox{orange!40}{\textbf{Soft Totals (Usable Ace)}}\\[1ex]
\begin{tabular}{|>{\raggedright\arraybackslash}p{3cm}|>{\raggedright\arraybackslash}p{3cm}|>{\raggedright\arraybackslash}p{2cm}|}
\hline
\textbf{Soft Total} & \textbf{Dealer Showing} & \textbf{Action} \\
\hline
  19--21 & Any & STICK \\
  18 & 2, 7, 8 & STICK \\
  18 & 3--6, 9, 10, A & HIT \\
  13--17 & Any & HIT \\
\hline
\end{tabular}
\vspace{2.3\baselineskip} % Adjust this to match the height of the left table
\end{minipage}

\vspace{0.5cm}

\noindent For {\bf task 2}, the goal is to try to 
imporve our win rate by using "basic" card counting
with a simple "Hi-Lo" system. The new action space is extended to include
the state of the game, which can be Low, Neutral, or High.
As mentioned in the
project description, we assume that we are playing with
a single deck of cards and the deck is reshuffled 
when 10 or less cards are left. A slight modification
is made due to this and we change states based on $>-3$ or 
$<3$ (not -5 and 5 that was initially suggested).\\

\noindent Now because the action space is bigger we train our agent for 1500000 episodes. After
evaluating for 1000000 episodes, for our bad luck, we do not observe
any improvement in the win rate.(maybe +0.5\% but is not constant).
This result aligns with expectations, while card counting is 
theoretically advantageous, its power lies primarily in adjusting
bet size based on the count, not just in modifying the HIT/STICK 
decisions. Since our game does not include betting, the 
potential of card counting cannot be fully realized in this phase.\\



\section*{Execution Details}

\noindent All the above are implemented in {\bf BlackJack\_PHASE\_1.ipynb} file.
There is a menu with the following options:

\begin{itemize}
    \item 1: Play a game manually against the dealer.
    \item 2: Evaluate random policy.
    \item 3: Evaluate threshold policy.
    \item 4: Train agent without card counting.
    \item 5: Evaluate the trainted agent without card counting.
    \item 6: Compare the trained agent (from option 2) with the optimal policy.
    \item 7: Train agent with card counting.
    \item 8: Evaluate the trainted agent with card counting.
    \item 9: Exit.
\end{itemize} 

\noindent To evaluate each policy (except the random ones), we need to train the agent first, otherwise Q-tables will be empty.



\end{document}